% Do not edit this file unless you know what are doing ;)
% The following line defines the type of printing paper and the size of the font along with the type of the file "article"
% It's possible to choose different document types check out the official latex documentations http://tobi.oetiker.ch/lshort/lshort.pdf
\documentclass[titlepage,a4paper,11pt]{article}

\usepackage[T1]{fontenc}

\usepackage[margin=1.0in]{geometry}

%defining authors
\author{Abdulrahman Alotaibi and Abdulaziz Alsaffar}
\title{KISR Development Constitution}

\begin{document}
\maketitle
\pagebreak
\thispagestyle{empty}
\tableofcontents
\pagebreak
\pagestyle{plain}
\setcounter{page}{1}
\pagenumbering{arabic}
\newpage

\section{Introduction}
This constitution aims to define the work follow of any development project that will come in the near future. We hope that it would make difference in the way you develop software. The universities have their rules of teaching, and most of them want to make students understand basic Algorithms and code syntax. This paper wants to give you hints and tast of real world developments with teams to produce programs that works for ever. Also, This paper will emphasis that programs produced by many programmers not a single coder that is setting in a basement somewhere.


With all that in mind, most of the tools in the next few sections would be divided into two main categories development tools and system/OS level tools. The development tools are those tools that you need to while developing a program/application. System level tools are programs that you need to do a specific job e.g. Web server or Database.

In the following sections we will descuse couple of tools and most of them are tools build on top of Unix like operating system. The following sections will put emphasis on the importance of using Free and Open source software. We admit that there is a learning curve and sometime a steep one, but after passing it things would make sense, and you will be a faster, more productive developer.

\section{Databases}
A database is system level tool. Most of databases run as a server on the system and you can access them by using the CLI, command line interface, or you can install GUI, graphical user interface, tool to interacte with the database. If you knew what you are doing, use the CLI or use the GUI.
\subsection{PostgreSQL}
PostgreSQL is a database that has a lot of features and it's used by many companies\cite{http://www.postgresql.org/about/users/} 
\subsubsection{PostGIS}
\subsection{MySQL}

\section{Testing Framework}
% no example here because it's a language dependent 

\section{Version Control}
\subsection{Git}
\subsubsection{Tutorial}

\section{Virtual Machine}
\subsection{VirtualBox}
\subsection{Virtual Machine Management Software}
\subsubsection{Vagrant}

\section{Documentation Software}
\subsection{Doxygen}
\subsection{Sphinx}

\section{Logging Software and Issue Tracking Software}
\subsection{example software}

\section{Webserver}
\subsection{Nginx}
\subsection{Apache}

\section{CGI vs WSGI}

\section{Useful commands}
\subsection{Regex}
\subsubsection{grep, egrep and fgrep}
\subsubsection{Sed}
\subsection{AWK and Unix Filters}
\subsection{Automake, make, Autoconfig, config, Cmake}
\subsection{ssh}
\subsection{ftp vs scp}


\section{IDEs}

\section{General programming advices}

\section{Package manegers}
\section{\latex{}}
\end{document}
