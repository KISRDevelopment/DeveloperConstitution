\documentclass[a4paper,11pt]{article}
\usepackage[T1]{fontenc}
\usepackage[margin=1.0in]{geometry}

\author{Abdulaziz Alsa{f}far and Abdulrahman Alotaibi}
\title{In-House Course Design Form}

\begin{document}
\maketitle
\pagestyle{plain}
\setcounter{page}{1}
\pagenumbering{arabic}

\section{Course Title:}
Full-Stack Software Development - Phase 2 - Team Projects

\section{Purpose:}
The purpose of this course is to walk participants through the process of contributing to an open-source project.
The tools and concepts taught in this course will allow participants to easily join and contribute to software development teams. It will also give prospective developers powerful tools for managing workflow.\\
Although the course uses open-source contribution to demonstrate key ideas, these concepts can be applied to any software development projects that require multiple developers.

\section{Course Objectives:}
\begin{itemize}
	\item Understand and use Version Control Systems
	\item Report and track issues and bugs
	\item Write tests for Test Driven Development
	\item Write patches
	\item Contribute to projects
\end{itemize}

\section{Course Topics:}
\begin{itemize}
	\item Version Control Systems - git
	\item Bug Reporting and Issue Tracking
	\item Logging software
	\item Test Driven Development
	\item Putting It All Together
\end{itemize}

\section{Training Methods:}
\begin{itemize}
	\item Demonstrations
	\item Practical Sessions
	\item Toy Project
\end{itemize}

\section{Participants:}
This course is for any programmer interested in working on a multi-developer project. 

\section{Prerequisites:}
Participants should have a coding and programming background. They should be familiar with basic developer tools taught in the course "Full-Stack Software Development - Phase 1 - Developer Tools". Participants should have their own laptops or a machine that they have administrative rights to.

\section{Course Date, Duration and Timings:}
Date: TODO\\
Duration: 1 Week\\
Timing: 9am - 12pm

\end{document}
