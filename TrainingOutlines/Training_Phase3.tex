\documentclass[a4paper,11pt]{article}
\usepackage[T1]{fontenc}
\usepackage[margin=1.0in]{geometry}

\author{Abdulaziz Alsa{f}far and Abdulrahman Alotaibi}
\title{In-House Course Design Form}

\begin{document}
\maketitle
\pagestyle{plain}
\setcounter{page}{1}
\pagenumbering{arabic}

\section{Course Title:}
Full-Stack Software Development - Phase 3 - Web Development

\section{Purpose:}
In this course participants will learn about backend development in terms of setting up webservers, PostgreSQL databases, and writing RESTful APIs. The course will focus on the Django framework. In the second part of the course, participants will work together as a team to develop a full-stack web app.

\section{Course Objectives:}
\begin{itemize}
	\item Use infrastructure-as-a-service (IaaS)
	\item Set up the production environment using Ansible
	\item Write RESTful APIs
	\item Build a web app using Django	
\end{itemize}

\section{Course Topics:}
\begin{itemize}
	\item Infrastructure-as-a-Service (IaaS)
	\item Ansible for automated deployment
	\item RESTful APIs
	\item Web Apps using Django
\end{itemize}

\section{Training Methods:}
\begin{itemize}
	\item Demonstrations
	\item Practical Sessions
	\item Toy Project
\end{itemize}

\section{Participants:}
This course is for programmers interested in full-stack web development. 

\section{Prerequisites:}
Participants should have a coding and programming background. They should be familiar with basic developer tools taught in the course "Full-Stack Software Development - Phase 1 - Developer Tools". They should be familiar with the team development concepts taught in the course "Full-Stack Software Development - Phase 2 - Team Projects". Participants should have their own laptops or a machine they have administrative rights to.

\section{Course Date, Duration and Timings:}
Date: TODO\\
Duration: 2 Weeks\\
Timing: 9am - 12pm

\end{document}
